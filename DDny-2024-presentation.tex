%\documentclass{beamer}
%
\documentclass[aspectratio=169]{beamer}
\setbeamertemplate{navigation symbols}{}

\usepackage{calc}
\usepackage{booktabs}
\usepackage{fp}
\usepackage{times}
\usepackage{xcolor}
\usepackage{graphicx}
%\usepackage{fourier}
%\usepackage[latin1]{inputenc}
\usepackage{tikz}
\usepackage{pgfplots}
\usepackage{animate}
\usepackage{ifthen}
\usepackage{todonotes}
\usepgfplotslibrary{fillbetween}
\usepackage{bm}
\usepackage{amsfonts}
\usepackage{amsmath}
\usepackage{fontspec}

\usetikzlibrary{arrows,automata,shapes,calc, patterns, backgrounds}
\usepackage{polyglossia}
\usepackage{algorithm}
\usepackage{algpseudocode}
\usepackage{listings}
\newcommand\user[2]{
	\begin{scope}[xshift=#1cm, yshift=#2cm]
		\clip (0, 0) circle (0.5);
		\fill[black!50] (0, 0) circle (0.5);
		\fill[white] (0, 0) circle (0.48);
		\fill[black!50] (0, -0.675) circle (0.4);
		\fill[black!50] (0, 0.075) circle (0.24);
  \end{scope}
}

\newcommand\ip[2]{
	\begin{scope}[xshift=#1cm, yshift=#2cm]
		%\rectangle[fill=black, rounded corners=0.2cm] (-0.5, -0.7) -- (0.5, 0.7);
		\draw[fill=black!50, thick, rounded corners=0.15cm] (-0.3, -0.5) rectangle (0.3, 0.5);
		\draw[fill=white] (0, -0.3) circle (0.1);
		\draw[fill=white, rounded corners=0.07cm] (-0.2, -0.1) rectangle (0.2, 0.0);
		\draw[fill=white, rounded corners=0.07cm] (-0.2, 0.1) rectangle (0.2, 0.2);
		\draw[fill=white, rounded corners=0.07cm] (-0.2, 0.3) rectangle (0.2, 0.4);
  \end{scope}
}

\newcommand\www[2]{
	\begin{scope}[xshift=#1cm, yshift=#2cm]
		\clip (0, 0) circle (0.5);
		\fill[black!50] (0, 0) circle (0.5);
		\fill[white] (-0.5, -0.175) rectangle (0.5,0.175);
		\node[text=black] at (0,0) {www};
  \end{scope}
}

\newcommand\pwww[2]{
	\begin{scope}[xshift=#1cm, yshift=#2cm]
		\clip (0, 0) circle (0.5);
		\fill[black!50] (0, 0) circle (0.5);
		\fill[white] (-0.5, -0.175) rectangle (0.5,0.175);
		\node[text=black] at (0,0) {\tiny{www}};
  \end{scope}
}

\newcommand\malwww[2]{
	\begin{scope}[xshift=#1cm, yshift=#2cm]
		\clip (0, 0) circle (0.5);
		\fill[red!65!black] (0, 0) circle (0.5);
		\fill[white] (-0.5, -0.175) rectangle (0.5,0.175);
		\node[text=yellow!65!black] at (0,0) {www};
  \end{scope}
}

\newcommand\pmalwww[2]{
	\begin{scope}[xshift=#1cm, yshift=#2cm]
		\clip (0, 0) circle (0.5);
		\fill[red!65!black] (0, 0) circle (0.5);
		\fill[white] (-0.5, -0.175) rectangle (0.5,0.175);
		\node[text=yellow!65!black] at (0,0) {\tiny{www}};
  \end{scope}
}

\newcommand\wwwline[3]{
	\node[circle, minimum size=1.1cm] (#3) at (0, #1) {};
	\www{0}{#1}
	\node[text width=50mm, align=right] at (-3.5, #1) {#2};
}

\newcommand\malwwwline[3]{
	\node[circle, minimum size=1.1cm] (#3) at (0, #1) {};
	\malwww{0}{#1}
	\node[text width=50mm, align=right] at (-3.5, #1) {#2};
}

\newcommand\userline[2]{
	\node[circle, minimum size=1.1cm] (#2) at (5, #1) {};
	\user{5}{#1}
	\node[text width=50mm, align=left] at (8.5, #1) {#2};
}

\newcommand\ipline[3]{
	\node[circle, minimum size=1.1cm] (#3) at (5, #1) {};
	\ip{5}{#1}
	\node[text width=50mm, align=left] at (8.5, #1) {#2};
}

\newcommand\lockicon[2]{
  \begin{scope}[xshift=#1cm, yshift=#2cm]
    \draw[fill=black!50]  (0.1, 0.9) -- (0.15, 0.9) -- (0.15, 1.2) to[out=90,in=180] (0.5, 1.6) to[out=0,in=90] (0.85, 1.2) -- (0.85, 0.9) -- (0.9, 0.9) to[out=0,in=90] (1, 0.8) -- (1,0.1) to[out=270,in=0] (0.9,0) -- (0.1,0) to[out=180,in=270] (0, 0.1) -- (0, 0.8) to[out=90,in=180] (0.1, 0.9);
    \fill[white] (0.5, 0.55) circle (0.1);
    \fill[white, rounded corners=0.5ex] (0.45, 0.3) rectangle (0.55, 0.6); 
    \draw[fill=white] (0.25, 0.9) -- (0.25, 1.2) to[out=90,in=180] (0.5, 1.5) to[out=0,in=90] (0.75, 1.2) -- (0.75, 0.9) -- (0.25, 0.9);
  \end{scope}
}

\newcommand\magnicon[2]{
  \begin{scope}[xshift=#1cm, yshift=#2cm]
    \draw[ultra thick]  (0, 0) circle (0.25);
    \draw[ultra thick] (225:0.25) -- (225:0.6);
  \end{scope}
}

\newcommand\envelope[2]{
  \begin{scope}[xshift=#1cm, yshift=#2cm]
    \draw[ultra thick, fill=white]  (-0.5, -0.3) rectangle (0.5, 0.3);
    \draw[ultra thick] (-0.5, 0.3) -- (0, -0.1) -- (0.5, 0.3);
  \end{scope}
}


\newcommand{\corr}{(\Letter)}
\newcommand{\name}[1]{\textit{#1}}
\newcommand{\mathfield}{\ensuremath{\mathbb}}
\newcommand{\mathmat}{\ensuremath{\mathbf}}
\newcommand{\mathset}{\ensuremath{\mathbb}}
\newcommand{\mathspace}{\ensuremath{\mathcal}}
\newcommand{\mathvec}{\ensuremath{\bm}}

\DeclareMathOperator*{\argmin}{arg\,min}
\DeclareMathOperator*{\argmax}{arg\,max}

\newcommand{\todoMD}[2][]{\todo[#1, color=purple!60!white, size=\small]{#2}}
\newcommand{\todoPP}[2][]{\todo[#1, color=blue!60!white, size=\small]{#2}}
\newcommand{\todoLB}[2][]{\todo[#1, color=red!60!white, size=\small]{#2}}

% notation
\newcommand{\method}{CSP}
\newcommand{\methodlong}{Convolutional Signal Propagation}
\newcommand{\U}{V} % set of nodes in hypergraph
\newcommand{\uu}{v} % node in hypergraph
\newcommand{\V}{E} % set of hyperedges in hypergraph
\newcommand{\vv}{e} % hyperedge in hypergraph
\newcommand{\HG}{\mathcal{H}} % hypergraph
\newcommand{\HH}{\mathmat{H}} % incidence matrix
\newcommand{\BG}{\mathcal{G}_\mathrm{bip}} % bipartite graph
\newcommand{\E}{E_\mathrm{bip}} % set of edges in bipartite graph
\newcommand{\D}{\mathmat{D}_v} % node degree matrix
\newcommand{\B}{\mathmat{D}_e} % hyperedge degree matrix
\newcommand{\X}{\mathmat{X}} % feature matrix
\newcommand{\Y}{\mathmat{Y}} % label matrix
\newcommand{\y}{y} % label
\newcommand{\x}{\mathvec{x}} % feature vector
\newcommand{\Tr}{V_\mathrm{train}} % Training set
\newcommand{\vdeg}{\mathrm{d}} % degree of node or edge
\newcommand{\edeg}{\delta} % degree of node or edge


\usepackage[backend = biber, style = iso-authoryear, sortlocale = en_US, autolang = other, bibencoding = UTF8]{biblatex}
\usepackage{datetime}
%\usepackage[utf8]{inputenc}
\usepackage{fontspec}
\usepackage{microtype}

\lstset{basicstyle={\small\ttfamily},
  belowskip=3mm,
  breakatwhitespace=true,
  breaklines=true,
  classoffset=0,
  columns=flexible,
  emptylines=10,
  framexleftmargin=0.25em,
  frameshape={}{}{}{}, %To remove to vertical lines on left, set `frameshape={}{}{}{}`
  keywordstyle=\color{blue},
  numbers=none, %If you want line numbers, set `numbers=left`
  numberstyle=\color{gray},
  showlines=true,
  showstringspaces=false,
  stringstyle=\color{mauve},
  tabsize=3,
  xleftmargin =1em
}

%\usetikzlibrary{external}
%\tikzexternalize[prefix=tikz/]

\makeatletter
\setbeamertemplate{footline}
{
  \leavevmode%
  \hbox{%
  \begin{beamercolorbox}[wd=.3\paperwidth,ht=2.25ex,dp=1ex,center]{author in head/foot}%
    \usebeamerfont{author in head/foot}Pavel Procházka et al.
  \end{beamercolorbox}%
  \begin{beamercolorbox}[wd=.45\paperwidth,ht=2.25ex,dp=1ex,center]{title in head/foot}%
    \usebeamerfont{title in head/foot}CSP: A Simple Scalable Algorithm for Hypergraphs
  \end{beamercolorbox}%
  \begin{beamercolorbox}[wd=.25\paperwidth,ht=2.25ex,dp=1ex,right]{date in head/foot}%
    \usebeamerfont{date in head/foot}\insertshortdate{}\hspace*{4em}
    \insertframenumber{} / \inserttotalframenumber\hspace*{2ex} 
  \end{beamercolorbox}}%
  \vskip0pt%
}

\makeatother

\def\ClusterGNN{\tiny{Chiang, Wei-Lin and Liu, Xuanqing and Si, Si and Li, Yang and Bengio, Samy and Hsieh, Cho-Jui.
  {\textcolor{blue}{"Cluster-gcn: An efficient algorithm for training deep and large graph convolutional networks"}}. {\it 25th ACM SIGKDD ,
 p. 257--266 2019}.}}

\title{Convolutional Signal Propagation: A Simple Scalable Algorithm for Hypergraphs}

\author{Pavel Procházka\inst{1}, \underline{Marek Dědič}\inst{1,2}, Lukáš Bajer\inst{1}}
    \institute{Cisco Systems, Inc. \and
    Faculty of Nuclear Sciences and Physical Engineering, Czech Technical University in Prague}


\date[DDny KM FJFI]{Doktorandské dny KM FJFI, 15. 11. 2024}

\begin{document}



 \frame{
    \titlepage
}

% 30 s

%\begin{frame}
%    \frametitle{Motivation -- Data Science in Real World (Cybersecurity) Setup}
%    \begin{itemize}
%        \item problem: identification/hunting malicious actors in networks
%        \item big (confidential) data
%        \item noisy labels possibly changing in time
%        \item heavily class imbalanced problems 
%        \item need for a simple flexible well scaling algorithm
%    \end{itemize}       
%\end{frame}

\begin{frame}
    \frametitle{Motivation -- Hunting for Malicious Entities in Network}
    \scalebox{0.8}{\begin{tikzpicture}
\node[cloud, cloud puffs=15.7, cloud ignores aspect, minimum width=16cm, minimum height=9cm, align=center, draw] (cloud) at (0cm, 0cm) {};
\begin{scope}[scale=0.75]

\uncover<3->{
\draw (7.0, 0.0) -- (7.0, 0.0);
\draw (7.0, 0.0) -- (6.0, -3.0);
\draw (7.0, 0.0) -- (4.0, 2.0);
\draw (7.0, 0.0) -- (5.0, 0.0);
\draw (7.0, 0.0) -- (7.0, 2.0);
\draw (7.0, 0.0) -- (4.0, 3.0);
\draw (-3.0, 3.0) -- (-3.0, 3.0);
\draw (-3.0, 3.0) -- (0.0, 5.0);
\draw (-3.0, 3.0) -- (-3.0, 2.0);
\draw (-3.0, 3.0) -- (-3.0, -1.0);
\draw (-3.0, 3.0) -- (-4.0, 3.0);
\draw (-3.0, 3.0) -- (0.0, 2.0);
\draw (-3.0, 3.0) -- (-6.0, 1.0);
\draw (-3.0, 3.0) -- (-4.0, -1.0);
\draw (-3.0, 3.0) -- (0.0, 2.0);
\draw (-3.0, 3.0) -- (1.0, 2.0);
\draw (-3.0, 3.0) -- (-4.0, 2.0);
\draw (-3.0, 3.0) -- (-5.0, -1.0);
\draw (0.0, 5.0) -- (-3.0, 3.0);
\draw (0.0, 5.0) -- (0.0, 5.0);
\draw (0.0, 5.0) -- (-3.0, 2.0);
\draw (0.0, 5.0) -- (2.0, 1.0);
\draw (0.0, 5.0) -- (-4.0, 3.0);
\draw (0.0, 5.0) -- (0.0, 2.0);
\draw (0.0, 5.0) -- (0.0, 2.0);
\draw (0.0, 5.0) -- (2.0, 3.0);
\draw (0.0, 5.0) -- (1.0, 2.0);
\draw (0.0, 5.0) -- (2.0, 4.0);
\draw (0.0, 5.0) -- (4.0, 3.0);
\draw (-3.0, 2.0) -- (-3.0, 3.0);
\draw (-3.0, 2.0) -- (0.0, 5.0);
\draw (-3.0, 2.0) -- (-3.0, 2.0);
\draw (-3.0, 2.0) -- (-3.0, -1.0);
\draw (-3.0, 2.0) -- (-4.0, 3.0);
\draw (-3.0, 2.0) -- (0.0, 2.0);
\draw (-3.0, 2.0) -- (-6.0, 1.0);
\draw (-3.0, 2.0) -- (-4.0, -1.0);
\draw (-3.0, 2.0) -- (0.0, -1.0);
\draw (-3.0, 2.0) -- (-1.0, -2.0);
\draw (-3.0, 2.0) -- (0.0, 2.0);
\draw (-3.0, 2.0) -- (1.0, 2.0);
\draw (-3.0, 2.0) -- (-4.0, 2.0);
\draw (-3.0, 2.0) -- (-5.0, -1.0);
\draw (-3.0, -1.0) -- (-3.0, 3.0);
\draw (-3.0, -1.0) -- (-3.0, 2.0);
\draw (-3.0, -1.0) -- (-3.0, -1.0);
\draw (-3.0, -1.0) -- (-3.0, -4.0);
\draw (-3.0, -1.0) -- (-7.0, -3.0);
\draw (-3.0, -1.0) -- (-4.0, 3.0);
\draw (-3.0, -1.0) -- (0.0, 2.0);
\draw (-3.0, -1.0) -- (-6.0, 1.0);
\draw (-3.0, -1.0) -- (-4.0, -1.0);
\draw (-3.0, -1.0) -- (0.0, -1.0);
\draw (-3.0, -1.0) -- (-1.0, -2.0);
\draw (-3.0, -1.0) -- (0.0, 2.0);
\draw (-3.0, -1.0) -- (-4.0, 2.0);
\draw (-3.0, -1.0) -- (1.0, -3.0);
\draw (-3.0, -1.0) -- (-5.0, -1.0);
\draw (-3.0, -4.0) -- (-3.0, -1.0);
\draw (-3.0, -4.0) -- (-3.0, -4.0);
\draw (-3.0, -4.0) -- (-7.0, -3.0);
\draw (-3.0, -4.0) -- (-4.0, -1.0);
\draw (-3.0, -4.0) -- (0.0, -1.0);
\draw (-3.0, -4.0) -- (-1.0, -2.0);
\draw (-3.0, -4.0) -- (1.0, -5.0);
\draw (-3.0, -4.0) -- (1.0, -4.0);
\draw (-3.0, -4.0) -- (1.0, -3.0);
\draw (-3.0, -4.0) -- (-5.0, -1.0);
\draw (2.0, 1.0) -- (0.0, 5.0);
\draw (2.0, 1.0) -- (2.0, 1.0);
\draw (2.0, 1.0) -- (0.0, 2.0);
\draw (2.0, 1.0) -- (0.0, -1.0);
\draw (2.0, 1.0) -- (-1.0, -2.0);
\draw (2.0, 1.0) -- (0.0, 2.0);
\draw (2.0, 1.0) -- (4.0, 2.0);
\draw (2.0, 1.0) -- (2.0, 3.0);
\draw (2.0, 1.0) -- (1.0, 2.0);
\draw (2.0, 1.0) -- (5.0, 0.0);
\draw (2.0, 1.0) -- (2.0, 4.0);
\draw (2.0, 1.0) -- (4.0, 3.0);
\draw (2.0, 1.0) -- (1.0, -3.0);
\draw (-7.0, -3.0) -- (-3.0, -1.0);
\draw (-7.0, -3.0) -- (-3.0, -4.0);
\draw (-7.0, -3.0) -- (-7.0, -3.0);
\draw (-7.0, -3.0) -- (-6.0, 1.0);
\draw (-7.0, -3.0) -- (-4.0, -1.0);
\draw (-7.0, -3.0) -- (-8.0, 0.0);
\draw (-7.0, -3.0) -- (-5.0, -1.0);
\draw (-4.0, 3.0) -- (-3.0, 3.0);
\draw (-4.0, 3.0) -- (0.0, 5.0);
\draw (-4.0, 3.0) -- (-3.0, 2.0);
\draw (-4.0, 3.0) -- (-3.0, -1.0);
\draw (-4.0, 3.0) -- (-4.0, 3.0);
\draw (-4.0, 3.0) -- (0.0, 2.0);
\draw (-4.0, 3.0) -- (-6.0, 1.0);
\draw (-4.0, 3.0) -- (-4.0, -1.0);
\draw (-4.0, 3.0) -- (0.0, 2.0);
\draw (-4.0, 3.0) -- (-4.0, 2.0);
\draw (-4.0, 3.0) -- (-5.0, -1.0);
\draw (0.0, 2.0) -- (-3.0, 3.0);
\draw (0.0, 2.0) -- (0.0, 5.0);
\draw (0.0, 2.0) -- (-3.0, 2.0);
\draw (0.0, 2.0) -- (-3.0, -1.0);
\draw (0.0, 2.0) -- (2.0, 1.0);
\draw (0.0, 2.0) -- (-4.0, 3.0);
\draw (0.0, 2.0) -- (0.0, 2.0);
\draw (0.0, 2.0) -- (0.0, -1.0);
\draw (0.0, 2.0) -- (-1.0, -2.0);
\draw (0.0, 2.0) -- (0.0, 2.0);
\draw (0.0, 2.0) -- (4.0, 2.0);
\draw (0.0, 2.0) -- (2.0, 3.0);
\draw (0.0, 2.0) -- (1.0, 2.0);
\draw (0.0, 2.0) -- (2.0, 4.0);
\draw (0.0, 2.0) -- (-4.0, 2.0);
\draw (0.0, 2.0) -- (4.0, 3.0);
\draw (-6.0, 1.0) -- (-3.0, 3.0);
\draw (-6.0, 1.0) -- (-3.0, 2.0);
\draw (-6.0, 1.0) -- (-3.0, -1.0);
\draw (-6.0, 1.0) -- (-7.0, -3.0);
\draw (-6.0, 1.0) -- (-4.0, 3.0);
\draw (-6.0, 1.0) -- (-6.0, 1.0);
\draw (-6.0, 1.0) -- (-4.0, -1.0);
\draw (-6.0, 1.0) -- (-8.0, 0.0);
\draw (-6.0, 1.0) -- (-4.0, 2.0);
\draw (-6.0, 1.0) -- (-5.0, -1.0);
\draw (-4.0, -1.0) -- (-3.0, 3.0);
\draw (-4.0, -1.0) -- (-3.0, 2.0);
\draw (-4.0, -1.0) -- (-3.0, -1.0);
\draw (-4.0, -1.0) -- (-3.0, -4.0);
\draw (-4.0, -1.0) -- (-7.0, -3.0);
\draw (-4.0, -1.0) -- (-4.0, 3.0);
\draw (-4.0, -1.0) -- (-6.0, 1.0);
\draw (-4.0, -1.0) -- (-4.0, -1.0);
\draw (-4.0, -1.0) -- (0.0, -1.0);
\draw (-4.0, -1.0) -- (-1.0, -2.0);
\draw (-4.0, -1.0) -- (-8.0, 0.0);
\draw (-4.0, -1.0) -- (-4.0, 2.0);
\draw (-4.0, -1.0) -- (-5.0, -1.0);
\draw (6.0, -3.0) -- (7.0, 0.0);
\draw (6.0, -3.0) -- (6.0, -3.0);
\draw (6.0, -3.0) -- (3.0, -4.0);
\draw (6.0, -3.0) -- (5.0, 0.0);
\draw (3.0, -4.0) -- (6.0, -3.0);
\draw (3.0, -4.0) -- (3.0, -4.0);
\draw (3.0, -4.0) -- (0.0, -1.0);
\draw (3.0, -4.0) -- (-1.0, -2.0);
\draw (3.0, -4.0) -- (5.0, 0.0);
\draw (3.0, -4.0) -- (1.0, -5.0);
\draw (3.0, -4.0) -- (1.0, -4.0);
\draw (3.0, -4.0) -- (1.0, -3.0);
\draw (0.0, -1.0) -- (-3.0, 2.0);
\draw (0.0, -1.0) -- (-3.0, -1.0);
\draw (0.0, -1.0) -- (-3.0, -4.0);
\draw (0.0, -1.0) -- (2.0, 1.0);
\draw (0.0, -1.0) -- (0.0, 2.0);
\draw (0.0, -1.0) -- (-4.0, -1.0);
\draw (0.0, -1.0) -- (3.0, -4.0);
\draw (0.0, -1.0) -- (0.0, -1.0);
\draw (0.0, -1.0) -- (-1.0, -2.0);
\draw (0.0, -1.0) -- (0.0, 2.0);
\draw (0.0, -1.0) -- (2.0, 3.0);
\draw (0.0, -1.0) -- (1.0, 2.0);
\draw (0.0, -1.0) -- (1.0, -5.0);
\draw (0.0, -1.0) -- (1.0, -4.0);
\draw (0.0, -1.0) -- (1.0, -3.0);
\draw (-1.0, -2.0) -- (-3.0, 2.0);
\draw (-1.0, -2.0) -- (-3.0, -1.0);
\draw (-1.0, -2.0) -- (-3.0, -4.0);
\draw (-1.0, -2.0) -- (2.0, 1.0);
\draw (-1.0, -2.0) -- (0.0, 2.0);
\draw (-1.0, -2.0) -- (-4.0, -1.0);
\draw (-1.0, -2.0) -- (3.0, -4.0);
\draw (-1.0, -2.0) -- (0.0, -1.0);
\draw (-1.0, -2.0) -- (-1.0, -2.0);
\draw (-1.0, -2.0) -- (0.0, 2.0);
\draw (-1.0, -2.0) -- (1.0, 2.0);
\draw (-1.0, -2.0) -- (1.0, -5.0);
\draw (-1.0, -2.0) -- (1.0, -4.0);
\draw (-1.0, -2.0) -- (1.0, -3.0);
\draw (-1.0, -2.0) -- (-5.0, -1.0);
\draw (-8.0, 0.0) -- (-7.0, -3.0);
\draw (-8.0, 0.0) -- (-6.0, 1.0);
\draw (-8.0, 0.0) -- (-4.0, -1.0);
\draw (-8.0, 0.0) -- (-8.0, 0.0);
\draw (-8.0, 0.0) -- (-4.0, 2.0);
\draw (-8.0, 0.0) -- (-5.0, -1.0);
\draw (0.0, 2.0) -- (-3.0, 3.0);
\draw (0.0, 2.0) -- (0.0, 5.0);
\draw (0.0, 2.0) -- (-3.0, 2.0);
\draw (0.0, 2.0) -- (-3.0, -1.0);
\draw (0.0, 2.0) -- (2.0, 1.0);
\draw (0.0, 2.0) -- (-4.0, 3.0);
\draw (0.0, 2.0) -- (0.0, 2.0);
\draw (0.0, 2.0) -- (0.0, -1.0);
\draw (0.0, 2.0) -- (-1.0, -2.0);
\draw (0.0, 2.0) -- (0.0, 2.0);
\draw (0.0, 2.0) -- (4.0, 2.0);
\draw (0.0, 2.0) -- (2.0, 3.0);
\draw (0.0, 2.0) -- (1.0, 2.0);
\draw (0.0, 2.0) -- (2.0, 4.0);
\draw (0.0, 2.0) -- (-4.0, 2.0);
\draw (0.0, 2.0) -- (4.0, 3.0);
\draw (4.0, 2.0) -- (7.0, 0.0);
\draw (4.0, 2.0) -- (2.0, 1.0);
\draw (4.0, 2.0) -- (0.0, 2.0);
\draw (4.0, 2.0) -- (0.0, 2.0);
\draw (4.0, 2.0) -- (4.0, 2.0);
\draw (4.0, 2.0) -- (2.0, 3.0);
\draw (4.0, 2.0) -- (1.0, 2.0);
\draw (4.0, 2.0) -- (5.0, 0.0);
\draw (4.0, 2.0) -- (7.0, 2.0);
\draw (4.0, 2.0) -- (2.0, 4.0);
\draw (4.0, 2.0) -- (4.0, 3.0);
\draw (2.0, 3.0) -- (0.0, 5.0);
\draw (2.0, 3.0) -- (2.0, 1.0);
\draw (2.0, 3.0) -- (0.0, 2.0);
\draw (2.0, 3.0) -- (0.0, -1.0);
\draw (2.0, 3.0) -- (0.0, 2.0);
\draw (2.0, 3.0) -- (4.0, 2.0);
\draw (2.0, 3.0) -- (2.0, 3.0);
\draw (2.0, 3.0) -- (1.0, 2.0);
\draw (2.0, 3.0) -- (5.0, 0.0);
\draw (2.0, 3.0) -- (2.0, 4.0);
\draw (2.0, 3.0) -- (4.0, 3.0);
\draw (1.0, 2.0) -- (-3.0, 3.0);
\draw (1.0, 2.0) -- (0.0, 5.0);
\draw (1.0, 2.0) -- (-3.0, 2.0);
\draw (1.0, 2.0) -- (2.0, 1.0);
\draw (1.0, 2.0) -- (0.0, 2.0);
\draw (1.0, 2.0) -- (0.0, -1.0);
\draw (1.0, 2.0) -- (-1.0, -2.0);
\draw (1.0, 2.0) -- (0.0, 2.0);
\draw (1.0, 2.0) -- (4.0, 2.0);
\draw (1.0, 2.0) -- (2.0, 3.0);
\draw (1.0, 2.0) -- (1.0, 2.0);
\draw (1.0, 2.0) -- (5.0, 0.0);
\draw (1.0, 2.0) -- (2.0, 4.0);
\draw (1.0, 2.0) -- (4.0, 3.0);
\draw (5.0, 0.0) -- (7.0, 0.0);
\draw (5.0, 0.0) -- (2.0, 1.0);
\draw (5.0, 0.0) -- (6.0, -3.0);
\draw (5.0, 0.0) -- (3.0, -4.0);
\draw (5.0, 0.0) -- (4.0, 2.0);
\draw (5.0, 0.0) -- (2.0, 3.0);
\draw (5.0, 0.0) -- (1.0, 2.0);
\draw (5.0, 0.0) -- (5.0, 0.0);
\draw (5.0, 0.0) -- (7.0, 2.0);
\draw (5.0, 0.0) -- (4.0, 3.0);
\draw (7.0, 2.0) -- (7.0, 0.0);
\draw (7.0, 2.0) -- (4.0, 2.0);
\draw (7.0, 2.0) -- (5.0, 0.0);
\draw (7.0, 2.0) -- (7.0, 2.0);
\draw (7.0, 2.0) -- (4.0, 3.0);
\draw (1.0, -5.0) -- (-3.0, -4.0);
\draw (1.0, -5.0) -- (3.0, -4.0);
\draw (1.0, -5.0) -- (0.0, -1.0);
\draw (1.0, -5.0) -- (-1.0, -2.0);
\draw (1.0, -5.0) -- (1.0, -5.0);
\draw (1.0, -5.0) -- (1.0, -4.0);
\draw (1.0, -5.0) -- (1.0, -3.0);
\draw (2.0, 4.0) -- (0.0, 5.0);
\draw (2.0, 4.0) -- (2.0, 1.0);
\draw (2.0, 4.0) -- (0.0, 2.0);
\draw (2.0, 4.0) -- (0.0, 2.0);
\draw (2.0, 4.0) -- (4.0, 2.0);
\draw (2.0, 4.0) -- (2.0, 3.0);
\draw (2.0, 4.0) -- (1.0, 2.0);
\draw (2.0, 4.0) -- (2.0, 4.0);
\draw (2.0, 4.0) -- (4.0, 3.0);
\draw (-4.0, 2.0) -- (-3.0, 3.0);
\draw (-4.0, 2.0) -- (-3.0, 2.0);
\draw (-4.0, 2.0) -- (-3.0, -1.0);
\draw (-4.0, 2.0) -- (-4.0, 3.0);
\draw (-4.0, 2.0) -- (0.0, 2.0);
\draw (-4.0, 2.0) -- (-6.0, 1.0);
\draw (-4.0, 2.0) -- (-4.0, -1.0);
\draw (-4.0, 2.0) -- (-8.0, 0.0);
\draw (-4.0, 2.0) -- (0.0, 2.0);
\draw (-4.0, 2.0) -- (-4.0, 2.0);
\draw (-4.0, 2.0) -- (-5.0, -1.0);
\draw (4.0, 3.0) -- (7.0, 0.0);
\draw (4.0, 3.0) -- (0.0, 5.0);
\draw (4.0, 3.0) -- (2.0, 1.0);
\draw (4.0, 3.0) -- (0.0, 2.0);
\draw (4.0, 3.0) -- (0.0, 2.0);
\draw (4.0, 3.0) -- (4.0, 2.0);
\draw (4.0, 3.0) -- (2.0, 3.0);
\draw (4.0, 3.0) -- (1.0, 2.0);
\draw (4.0, 3.0) -- (5.0, 0.0);
\draw (4.0, 3.0) -- (7.0, 2.0);
\draw (4.0, 3.0) -- (2.0, 4.0);
\draw (4.0, 3.0) -- (4.0, 3.0);
\draw (1.0, -4.0) -- (-3.0, -4.0);
\draw (1.0, -4.0) -- (3.0, -4.0);
\draw (1.0, -4.0) -- (0.0, -1.0);
\draw (1.0, -4.0) -- (-1.0, -2.0);
\draw (1.0, -4.0) -- (1.0, -5.0);
\draw (1.0, -4.0) -- (1.0, -4.0);
\draw (1.0, -4.0) -- (1.0, -3.0);
\draw (1.0, -3.0) -- (-3.0, -1.0);
\draw (1.0, -3.0) -- (-3.0, -4.0);
\draw (1.0, -3.0) -- (2.0, 1.0);
\draw (1.0, -3.0) -- (3.0, -4.0);
\draw (1.0, -3.0) -- (0.0, -1.0);
\draw (1.0, -3.0) -- (-1.0, -2.0);
\draw (1.0, -3.0) -- (1.0, -5.0);
\draw (1.0, -3.0) -- (1.0, -4.0);
\draw (1.0, -3.0) -- (1.0, -3.0);
\draw (-5.0, -1.0) -- (-3.0, 3.0);
\draw (-5.0, -1.0) -- (-3.0, 2.0);
\draw (-5.0, -1.0) -- (-3.0, -1.0);
\draw (-5.0, -1.0) -- (-3.0, -4.0);
\draw (-5.0, -1.0) -- (-7.0, -3.0);
\draw (-5.0, -1.0) -- (-4.0, 3.0);
\draw (-5.0, -1.0) -- (-6.0, 1.0);
\draw (-5.0, -1.0) -- (-4.0, -1.0);
\draw (-5.0, -1.0) -- (-1.0, -2.0);
\draw (-5.0, -1.0) -- (-8.0, 0.0);
\draw (-5.0, -1.0) -- (-4.0, 2.0);
\draw (-5.0, -1.0) -- (-5.0, -1.0);
\draw (-5, 4.0) -- (-4.0, 2.0);
\draw (-7, 3.0) -- (-4.0, 2.0);
\draw (-5, 4.0) -- (-4.0, 3.0);
\draw (-7, 3.0) -- (-8.0, 0);
}



\user{7.0}{0};
\ip{-3.0}{3.0};
\www{0.0}{5.0};
\envelope{-3.0}{2.0};
\user{-3.0}{-1.0};
\ip{-3.0}{-4.0};
\www{2.0}{1.0};
\envelope{-7.0}{-3.0};
\user{-4.0}{3.0};
\ip{0.0}{2.0};
\www{-6.0}{1.0};
\envelope{-4.0}{-1.0};
\user{6.0}{-3.0};
\ip{3.0}{-4.0};
\www{0.0}{-1.0};
\envelope{-1.0}{-2.0};
\user{-8.0}{0.0};
\ip{0.0}{2.0};
\www{4.0}{2.0};
\envelope{2.0}{3.0};
\user{1.0}{2.0};
\ip{5.0}{0.0};
\www{7.0}{2.0};
\envelope{1.0}{-5.0};
\user{2.0}{4.0};
\ip{-4.0}{2.0};
\www{4.0}{3.0};
\envelope{1.0}{-4.0};
\user{1.0}{-3.0};
\ip{-5.0}{-1.0};
\www{-5}{4};
\www{-7}{3};
\envelope{0}{0};


\node[fill=white, rectangle] at (9, 5) {\LARGE{\~{}100M entities}};
\uncover<3->{
\node[fill=white, rectangle] at (-9, -4.5) {\LARGE{\~{}100B records per hour}};
}
\uncover<2->{
\node[fill=white, rectangle] at (-6, 5.5) {\LARGE{\~{}100 malicious entities}};
\malwww{-5}{4};
\malwww{-7}{3};
}
\uncover<4->{
\node[fill=white, rectangle] at (0, -6) {\LARGE{confidential data}};
}
\uncover<5->{
\node[fill=white, rectangle] at (8, -5) {\LARGE{expensive evaluation}};
}
\uncover<6>{
\node[fill=white, rectangle, inner sep=1cm, opacity=0.8] at (0, 0) {\huge{\alert{Need for a simple flexible efficient retrieval algorithm}}};
}
%\lockicon{-3}{-3}{0.02};
%\magnicon{-2}{1};
\end{scope}
\end{tikzpicture} }    
\end{frame}

% 1.5 min

\begin{frame}
    \frametitle{Convolutional Signal Propagation - Problem to Solve}
    \begin{columns}    
    \column{0.6\linewidth}
    \begin{itemize}
        \item Telemetry data
        \item Positive domains: {\it{evil.com}}, {\it{ransomware.com}}
        \item Unknown domains: other
        \item<2-> \alert{Which of the unknown domains are positive?}
        \item<3-> Tasks:
        \begin{itemize}
            \item Domain \emph{reputation} (score)
            \item Domain \emph{retrieval} (rank)
            \item Domain binary \emph{classification}
        \end{itemize} 
    \end{itemize}
    \column{0.3\linewidth}
    \small
    \begin{tabular}{ll}
    \toprule
    Domain & User \\
    \midrule
    candidate.com & Alice \\
    candidate.com & Charlie \\
    evil.com & Alice \\
    evil.com & Bob \\
    evil.com & David \\
    ransomware.com & Alice \\
    ransomware.com & Bob \\
    somedomain.ch & Bob \\
    somedomain.ch & Charlie \\
    unknown.com & Alice \\
    unknown.com & David \\
    \bottomrule
    \end{tabular}    
    \end{columns}
\end{frame}

% 3.5 min

\begin{frame}
    \frametitle{Convolutional Signal Propagation - Data Representation}
    \scalebox{0.8}{
    \begin{tikzpicture}
\tikzset{node/.style={draw,circle}}
\tikzset{edge/.style={draw,rectangle,inner sep=2pt}}

\tikzset{nodepos/.style={draw,circle, fill=red!20!white}}

%% Bipartite Graph
\begin{scope}[node distance=1.5cm, circle, draw=black]
	\node[nodepos] at (0, -2) (x1) {$\uu_1$};
    \node[nodepos, below of=x1] (x2) {$\uu_2$};
    \node[node, below of=x2] (x3) {$\uu_3$};
    \node[node, below of=x3] (x4) {$\uu_4$};
    \node[node, below of=x4] (x5) {$\uu_5$};
\end{scope}
    
\begin{scope}[node distance=2cm]
    \node[node] at (3, -2) (e1) {$\vv_1$};
    \node[node, below of=e1] (e2) {$\vv_2$};
    \node[node, below of=e2] (e3) {$\vv_3$};
    \node[node, below of=e3] (e4) {$\vv_4$};
\end{scope}

  \draw (e1) -- (x3);
  \draw (e1) -- (x1);
  \draw (e1) -- (x2);
  \draw (e2) -- (x2);
  \draw (e2) -- (x1);
  \draw (e4) -- (x1);
  \draw (e1) -- (x5);
  \draw (e2) -- (x4);
  \draw (e4) -- (x3);
  \draw (e3) -- (x4);
  \draw (e3) -- (x5);
  
%% Hyper-Graph

\begin{scope}[xshift=7cm, yshift=-5cm]
    \node[nodepos] at (0:2) (xx2) {$\uu_2$};
    \node[nodepos] at (72:2)(xx1) {$\uu_1$};
    \node[node] at (144:2)(xx3) {$\uu_3$};
    \node[node] at (216:2)(xx5) {$\uu_5$};
    \node[node] at (288:2)(xx4) {$\uu_4$};

    %\node[edge] at (40:0.65) (ee2) {$e_2$};
    %\node[edge] at (105:0.65)(ee4) {$e_4$};
    %\node[edge] at (255:0.6)(ee3) {$e_3$};
    %\node[edge] at (0:0) (ee1) {$e_1$};
\end{scope}

  %\draw (ee1) -- (xx3);
  %\draw (ee1) -- (xx1);
  %\draw (ee1) -- (xx2);
  %\draw (ee2) -- (xx2);
  %\draw (ee2) -- (xx1);
  %\draw (ee4) -- (xx1);
  %\draw (ee1) -- (xx5);
  %\draw (ee2) -- (xx4);
  %\draw (ee4) -- (xx3);
  %\draw (ee3) -- (xx4);
  %\draw (ee3) -- (xx5);

\begin{scope}[fill opacity=0.2]
    \filldraw[fill=yellow!20] ($(xx1)+(0.5,0)$) 
        to[out=90,in=0] ($(xx1) + (0,0.5)$) 
        to[out=180,in=70] ($(xx3) + (-0.5,0)$) 
        to[out=250,in=210] ($(xx3) + (0,-0.5)$)
        to[out=30,in=270] ($(xx1) + (0.5,0)$);
    \filldraw[fill=green!20] ($(xx5)+(-0.5,0)$) 
        to[out=250,in=180] ($(xx5) + (0,-0.5)$) 
        to[out=0,in=210] ($(xx4) + (0,-0.5)$) 
        to[out=30,in=290] ($(xx4) + (0.5, 0)$)
        to[out=110,in=70] ($(xx5) + (-0.5,0)$);
    \filldraw[fill=cyan!20] ($(xx1)+(-0.5,0)$) 
        to[out=270,in=90] ($(xx4) + (-0.5,0)$) 
        to[out=270,in=180] ($(xx4) + (0,-0.5)$) 
        to[out=0,in=270] ($(xx2) + (0.5,0)$) 
        to[out=90,in=0] ($(xx1) + (0, 0.5)$)
        to[out=180,in=90] ($(xx1) + (-0.5, 0)$);
    \filldraw[fill=gray!20] ($(xx1)+(0,0.5)$) 
        to[out=180,in=90] ($(xx3) + (-0.5,0)$) 
        to[out=270,in=210] ($(xx5) + (0,-0.5)$) 
        to[out=30,in=270] ($(xx2) + (0.5,0)$) 
        to[out=90,in=0] ($(xx1) + (0, 0.5)$);
\end{scope}


\begin{scope}[xshift=13cm, yshift=-5cm]
\node at (0,0) {
    $\HH=\left[\begin{array}{cccc}
 1 & 1 & 0 & 1 \\
 1 & 1 & 0 & 0 \\
 1 & 0 & 0 & 1 \\
 0 & 1 & 1 & 0 \\
 1 & 0 & 1 & 0 
     \end{array}
     \right]$, $\mathmat{Y}=\left[\begin{array}{c}
 1 \\
 1 \\
 0 \\
 0 \\
 0 
     \end{array}
     \right]$
};
\end{scope}

\end{tikzpicture}
    }
\end{frame}

% 5 min

\begin{frame}
    \frametitle{Convolutional Signal Propagation - Algorithm Description}
    \scalebox{0.8}{
    \input{images/csp_algorithm}
    }
\end{frame}

% 6 min

\begin{frame}
    \frametitle{Convolutional Signal Propagation - Algorithm Formulation}
    \begin{itemize}
    \item Basic form
        \begin{equation*}\hspace*{-3cm}
            \uncover<2>{x_k^{(l+1)} = \frac{1}{deg\left( u_k \right)}\sum_{\substack{j \\ u_k \in v_j}}} \frac{1}{deg \left( v_j \right)}\sum_{\substack{i \\ u_i \in v_j}} x_i^{(l)}
        \end{equation*}
    \item Matrix form    
        \begin{equation*}\hspace*{-3cm}
        \uncover<2>{\X^{(l+1)} = \D^{-1} \HH} \B^{-1} \HH^T \X^{(l)}
        \end{equation*}
    \item SQL implementation (1 layer)
    \begin{onlyenv}<1>
        \begin{lstlisting}[language=SQL,basicstyle=\small]


    SELECT nodeId, AVG(X) OVER (PARTITION BY edgeId) AS my FROM graph


        \end{lstlisting}
    \end{onlyenv}
    \begin{onlyenv}<2>
        \begin{lstlisting}[language=SQL,basicstyle=\small]
SELECT nodeId, AVG(my) AS score
FROM (
    SELECT nodeId, AVG(X) OVER (PARTITION BY edgeId) AS my FROM graph
)
GROUP BY nodeId;
        \end{lstlisting}
    \end{onlyenv}
\end{itemize}


\begin{tikzpicture}[remember picture,overlay]
  \tikzset{shift={(current page.center)},yshift=5cm, xshift=4.5cm}
  \tikzset{node/.style={draw,circle}}
  \tikzset{edge/.style={draw,rectangle,inner sep=2pt}}

\tikzset{nodepos/.style={draw,circle, fill=red!20!white}}
\tikzset{nodestress/.style={draw,circle, ultra thick}}
\uncover<2>{
\begin{scope}[node distance=1cm, circle, draw=black]
	\node[nodepos, opacity=0.2] at  (0, -2) (x1) {$\uu_1$};
    \node[nodepos, opacity=0.2, below of=x1] (x2) {$\uu_2$};
    \node[nodestress, below of=x2] (x3) {$\uu_3$};
    \node[node, opacity=0.2, below of=x3] (x4) {$\uu_4$};
    \node[node, opacity=0.2, below of=x4] (x5) {$\uu_5$};
\end{scope}
    
\begin{scope}[node distance=1.2cm]
    \node[node] at (1.8,-2) (e1) {$\vv_1$};
    \node[node, below of=e1, opacity=0.2] (e2) {$\vv_2$};
    \node[node, below of=e2, opacity=0.2] (e3) {$\vv_3$};
    \node[node, below of=e3] (e4) {$\vv_4$};
\end{scope}

  \draw (e1) -- (x3);
  \draw[opacity=0.2] (e1) -- (x1);
  \draw[opacity=0.2] (e1) -- (x2);
  \draw[opacity=0.2] (e2) -- (x2);
  \draw[opacity=0.2] (e2) -- (x1);
  \draw[opacity=0.2] (e4) -- (x1);
  \draw[opacity=0.2] (e1) -- (x5);
  \draw[opacity=0.2] (e2) -- (x4);
  \draw (e4) -- (x3);
  \draw[opacity=0.2] (e3) -- (x4);
  \draw[opacity=0.2] (e3) -- (x5);
}

\uncover<1>{
\begin{scope}[node distance=1cm, circle, draw=black]
    \node[nodepos] at (0, -2) (x1) {$\uu_1$};
    \node[nodepos, below of=x1] (x2) {$\uu_2$};
    \node[node, below of=x2] (x3) {$\uu_3$};
    \node[node, opacity=0.2, below of=x3] (x4) {$\uu_4$};
    \node[node, below of=x4] (x5) {$\uu_5$};
\end{scope}
    
\begin{scope}[node distance=1.2cm]
    \node[nodestress] at (1.8,-2)  (e1) {$\vv_1$};
    \node[node, below of=e1, opacity=0.2] (e2) {$\vv_2$};
    \node[node, below of=e2, opacity=0.2] (e3) {$\vv_3$};
    \node[nodestress, below of=e3] (e4) {$\vv_4$};
\end{scope}

  \draw (e1) -- (x3);
  \draw (e1) -- (x1);
  \draw (e1) -- (x2);
  \draw[opacity=0.2] (e2) -- (x2);
  \draw[opacity=0.2] (e2) -- (x1);
  \draw (e4) -- (x1);
  \draw (e1) -- (x5);
  \draw[opacity=0.2] (e2) -- (x4);
  \draw (e4) -- (x3);
  \draw[opacity=0.2] (e3) -- (x4);
  \draw[opacity=0.2] (e3) -- (x5);
}
\end{tikzpicture}
   
\end{frame}

% 7 min

\begin{frame}
    \frametitle{Convolutional Signal Propagation - Discussion}
    \begin{itemize}
        \item Method suitable for malicious entity retrieval 
        \item<2-> RQ1: How is CSP related to the other methods?
        \item<2-> RQ2: Where can CSP be applied? 
        \item<2-> RQ3: How does CSP perform compared to standard reference methods?
        \item<3-> When should I consider to use CSP and why?
    \end{itemize}
\end{frame}

% 8.5 min

\begin{frame}
    \frametitle{CSP Comparison with Other Methods}
			\begin{itemize}
                    \item CSP: ${\X^{(l+1)} = \D^{-1} \HH} \B^{-1} \HH^T \X^{(l)}$
                    \item<2-> Naive Bayes
                    \begin{itemize}
                        \item Fitting binomial Naive Bayes in the 1st step ($\B^{-1} \HH^T \X^{(l)}$)
                        \item Simple mean instead of product and Bayes rule in the 2nd step
                    \end{itemize}
			    \item<3-> Hyper-Graph Convolutional Network (HGCN)
                        \begin{equation*}
            \X^{(l+1)} = \sigma(\D^{−1} \HH \mathmat{W} \B^{−1} \HH^T \X^{(l)} \boldsymbol{\Theta}),
        \end{equation*}
         where \( \mathmat{W} \) and \( \boldsymbol{\Theta} \) are realized as non-learnable identity matrices.                    
        			\end{itemize}
\end{frame}

\begin{frame}
    \frametitle{CSP Comparison with Label Propagation}
			\begin{itemize}
                    \item CSP: ${\X^{(l+1)} = \D^{-1} \HH} \B^{-1} \HH^T \X^{(l)}$
                    \item<2-> Label Propagation
                    \begin{equation*} 
                        \Y^{(l+1)} = \alpha \D^{−1} \mathmat{A} \Y^{(l)} + \left( 1 - \alpha \right) \Y^{(l)}
                    \end{equation*}
                     with $\HH \B^{-1} \HH^T = \frac{1}{2}( \mathmat{A} + \D)$,
                    \begin{equation*}
                        \X^{(l+1)} = \frac{1}{2} \D^{-1} \mathmat{A} \X^{(l)} + \frac{1}{2} \X^{(l)}.
                    \end{equation*}			    
			\end{itemize}
\end{frame}

% 10 min

\begin{frame}
    \frametitle{Experiment Setup - Datasets}
    \scalebox{0.82}{
        \begin{tabular}{llllllll}
        \toprule
        \textbf{Dataset} & \textbf{Node} & \textbf{Node label} & \textbf{Hyper-edge} & \textbf{\#nodes} & \textbf{\#hyper-edges} & \textbf{\#bip.\ edges} & \textbf{\#classes} \\
        \midrule
        Cora-CA & paper & topic & author & 2708 & 1072  & 4585 & 7\\
        Cora-CC & paper & topic & co-cited papers & 2708 & 1579  & 4786 &7\\
        CiteSeer & paper & topic & co-cited papers& 3312 & 1079  & 3453 &6\\
        DBLP & paper & topic & author& 41302 & 22363 & 99561 & 6\\
        PubMed & paper & topic & co-cited papers& 19717 & 7963  & 34629 &3\\
        Corona & text & sentiment & token& 44955 & 998  & 3455918 &5\\
        movie-RA & movie & category & user& 62423 & 162541 & 25000095 &20\\
        movie-TA & movie & category & tag& 62423 & 14592  & 1093360 &20\\
        \bottomrule
        \end{tabular}
        }
\end{frame}

% 11.5 min

\begin{frame}
    \frametitle{Experiment Setup}
    \begin{itemize}
        \item Tasks related to each class:
        \begin{itemize}
            \item Retrieval (10\% train 90\% test split)
            \item Binary classification (90\% train 10\% test split)
        \end{itemize}
        \item Considered methods:
        \begin{itemize}
            \item Convolutional Signal Propagation (1,2 and 3 layers)
            \item Naive Bayes operating with one hot features
            \item Random forest, logistic regression, HGCN with NMF
        \end{itemize}
        \item LOOCV with 10 random folds
    \end{itemize}
\end{frame}

% 12.5 min

\begin{frame}
\frametitle{Average ROC-AUC for Binary Classification Task}
\scalebox{0.75}{
    \begin{tabular}{lrrrrrrrr}
        \toprule
        \textbf{Method} & \textbf{CiteSeer} & \textbf{Cora-CA} & \textbf{Cora-CC} & \textbf{DBLP} & \textbf{PubMed} & \textbf{Corona} & \textbf{movie-RA} & \textbf{movie-TA} \\
        \midrule
        \textbf{\method{} 1-layer} & \underline{0.646} & \underline{0.882} & 0.716 & \underline{0.968} & 0.537 & \textbf{0.704} & \underline{0.789} & \underline{0.717} \\
        \textbf{\method{} 2-layer} & 0.630 & \underline{0.872} & 0.686 & \underline{0.972} & 0.518 & 0.618 & 0.700 & \underline{0.697} \\       
        \textbf{\method{} 3-layer} & 0.613 & 0.862 & 0.655 & \underline{0.972} & 0.516 & 0.580 & 0.640 & 0.673 \\
        \textbf{Naive Bayes} & \textbf{0.686} & \textbf{0.913} & \underline{0.775} & \textbf{0.974} & \textbf{0.633} & \textbf{0.704} & \underline{0.753} & 0.557 \\   
        \textbf{HGCN-NMF} & \underline{0.659} & 0.832 & \textbf{0.786} & 0.775 & \underline{0.624} & 0.622 & \underline{0.794} & \textbf{0.724} \\
        \textbf{LR-NMF} & 0.604 & 0.794 & 0.703 & 0.705 & 0.556 & 0.647 & \underline{0.754} & \underline{0.675} \\
        \textbf{RF-NMF} & \underline{0.667} & \underline{0.897} & \underline{0.772} & 0.905 & \underline{0.623} & 0.617 & \textbf{0.797} & \underline{0.691} \\
        \textbf{Random} & 0.499 & 0.505 & 0.489 & 0.501 & 0.502 & 0.503 & 0.499 & 0.487 \\
        \bottomrule
    \end{tabular}
}
\end{frame}

% 13 min

\begin{frame}
\frametitle{Average P@100 for Retrieval Task}
\scalebox{0.75}{
    \begin{tabular}{lrrrrrrrr}
        \toprule
        \textbf{Method} & \textbf{CiteSeer} & \textbf{Cora-CA} & \textbf{Cora-CC} & \textbf{DBLP} & \textbf{PubMed} & \textbf{Corona} &\textbf{movie-RA} & \textbf{movie-TA} \\
        \midrule
        \textbf{\method{} 1-layer} & 0.494 & \underline{0.703} & 0.530 & 0.869 & 0.798 & \textbf{0.530} & 0.334 & 0.156 \\
        \textbf{\method{} 2-layer} & \underline{0.558} & \underline{0.718} & \underline{0.681} & 0.865 & \underline{0.826} & 0.440 & 0.336 & 0.186 \\       
        \textbf{\method{} 3-layer} & \textbf{0.568} & \textbf{0.721} & \textbf{0.707} & 0.869 & \underline{0.850} & 0.332 & 0.238 & 0.186 \\       
        \textbf{Naive Bayes} & 0.471 & \underline{0.686} & 0.491 & \textbf{0.951} & \underline{0.860} & 0.446 & 0.216 & 0.153 \\
        \textbf{HGCN-NMF} & 0.482 & \underline{0.671} & 0.607 & 0.794 & \textbf{0.871} & 0.392 & 0.257 & 0.148 \\
        \textbf{LR-NMF} & 0.329 & 0.603 & 0.372 & 0.602 & 0.735 & 0.397 & \textbf{0.580} & \textbf{0.356} \\
        \textbf{RF-NMF} & 0.303 & 0.474 & 0.482 & 0.843 & 0.794 & 0.381 & 0.470 & 0.131 \\
        \textbf{Random} & 0.153 & 0.132 & 0.129 & 0.155 & 0.308 & 0.180 & 0.040 & 0.055 \\
        \bottomrule
    \end{tabular}
}
\end{frame}

% 13.5 min

\begin{frame}
\frametitle{Execution Time in Microseconds}
\scalebox{0.75}{
     \begin{tabular}{lrrrrrrrr}
        \toprule
        \textbf{Method} & \textbf{CiteSeer} & \textbf{Cora-CA} & \textbf{Cora-CC} & \textbf{DBLP} & \textbf{PubMed} & \textbf{Corona} & \textbf{movie-RA} & \textbf{movie-TA} \\
        \midrule
        \textbf{\method{} 1-layer} & \textbf{1.35} & \textbf{1.23} & \textbf{1.24} & \textbf{3.51} & \textbf{4.01} & \textbf{22.83} & 170.63 & \textbf{12.07} \\             
        \textbf{\method{} 2-layer} & 2.41 & 2.25 & 2.24 & 7.46 & 7.35 & 47.39 & 349.67 & 25.88 \\
        \textbf{\method{} 3-layer} & 3.31 & 3.09 & 3.08 & 9.65 & 9.1 & 70.98 & 506.1 & 35.75 \\
        \textbf{Naive Bayes} & 2.59 & 2.73 & 2.54 & 11.16 & 5.35 & 89.3 & 1 051.53 & 35.61 \\
        \textbf{*HGCN-NMF} & 23 714 & 23 690 & 23 709 & 29 123 & 23 879 & 112 314 & 620 717 & 70 778 \\
        \textbf{*LR-NMF} & 44.45 & 51.59 & 49.54 & 64.96 & 44.69 & 56 & \textbf{61.82} & 74.07 \\
        \textbf{*RF-NMF} & 140.76 & 143.85 & 134.52 & 1 148.83 & 323.6 & 1 608.58 & 697.85 & 716.68 \\
        \bottomrule
    \end{tabular} 
    }    
    
    *NMF processing is not included in evaluation of the time duration
\end{frame}

% 14 min

\begin{frame}
\frametitle{Conclusions}
    \begin{itemize}
        \item Method suitable for malicious entity retrieval 
        \item<2-> RQ1: How is CSP related to the other methods?
        \begin{itemize}
            \item Alternative way of NB inference
            \item Extension of label propagation for hypergraphs
            \item special case of forward pass of HGCN
        \end{itemize}
        \item<3-> RQ2: Where can CSP be applied? 
        Learning method for classification and retrieval on graphs, hyper-graphs or problems described by one-hot features.       
        \item<4-> RQ3: How does CSP perform compared to standard reference methods? Comparable performance while being significantly cheaper to compute.
        \item<5> When should I consider to use CSP and why?
    \end{itemize}
\end{frame}

% 15 min

\end{document}
